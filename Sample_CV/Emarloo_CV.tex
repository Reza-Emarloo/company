

\documentclass[11pt,a4paper]{bidimoderncv}
% M.Amintoosi
% برای اجرا باید دنباله کارهای زیر را انجام دهید:
% xelatex Amintoosi_CV
% bibtex Amintoosi_CV
% xelatex Amintoosi_CV
% xelatex Amintoosi_CV

\usepackage[numbers]{natbib}%
%\setlength{\bibhang}{2em}

\cvtheme[blue]{bidiclassic}%casual} 
\usepackage{pifont}
%\usepackage[scale=0.8]{geometry}
\usepackage{xepersian}
\settextfont[Scale=1]{XB Yas}%{XB Niloofar}%{B Nazanin}%
\setlatintextfont[Scale=1]{Times New Roman}%{Times New Roman}%
%\defpersianfont\nastaliq[Scale=1.2]{IranNastaliq}

\AtBeginDocument{\recomputelengths} 
\firstname{رضا }
\familyname{عمارلو‌}
\resumename{رزومه}
\title{شرح حال}               
%\address{نیشابور}   
\mobile{09363639239}
%\phone{۰5143773345}  
%\fax{شماره فکس}                          
\email{emarloreza2@gmail.com}
%\extrainfo{\httplink[https://www.linkedin.com/in/reza-emarloo/]{www.linkedin.com/in/reza-emarloo/}} 
%\photo[64pt]{rsz_120201107_102352.jpg}                         
%\quote{نقل قول}  
\begin{document}
	\maketitle
	\section{وضعیت فعلی}
	\cventry{}{‌دارای کارت پایان خدمت}{}{  }{}{}
	\cventry{}{مجرد}{}{ }{}{ }  
	\cventry{}{کارشناس تحلیل سبد گردان هامرز}{}{ }{}{ } 
	 
	\section{تحصیلات}
	\cventry{}{کارشناسی }{}{ }{}{ ریاضی(دانشگاه بیرجند)}    
	\cventry{}{کارشناسی ارشد}{}{ }{}{مالی(مهندسی مالی و مدیریت ریسک) دانشگاه اصفهان } 
	\cventry{پایان‌نامه}{محاسبه ریسک اعتباری بانک‌ها با رویکرد تحلیل بقا }{با درجه عالی دفاع شده است}{ }{}{}
	    
\section{مدارک}
	\cventry{}{دارای مدرک اصول بازار سرمایه}{}{ }{}{ } 
	\cventry{}{دارای مدرک تحلیلگری بازار سرمایه}{}{ }{}{ } 
	  	
	\section{مهارت‌ها }

	
	\cvlistdoubleitem[\ding{72}\ding{72}\ding{72}\ding{73}]{ تسلط بر تحلیل بنیادی صنایع(DDM,DCF,P/E)}{تسلط بر تجزیه و تحلیل صورت‌های مالی }
	
	\cvlistdoubleitem[\ding{72}\ding{72}\ding{72}\ding{73}]{آشنایی با انواع ابزار‌های مالی }{توانایی تحلیل اقتصاد کلان و بازار‌های جهانی}
	
	\cvlistdoubleitem[\ding{72}\ding{72}\ding{78}\ding{73}]{‌ تسلط نسبی بر زبان انگلیسی}{آشنایی با مدیریت ریسک \lr{\small(Risk Management)}}
	
	\cvlistdoubleitem[\ding{72}\ding{72}\ding{78}\ding{73}]{تسلط بر مجموعه آفیس}{مسلط به نرم‌افزار لاتک \lr{(\LaTeX)}}%\lr{\small Seam Carving}}
%%%%%%%%%%%%%%%%%%%%%%%%%%%%%%%%%%%%%%%%%%%%%%%%	
\section{مهارت‌های کمی}
\cvlistdoubleitem[\ding{72}\ding{72}\ding{72}\ding{73}]{ تسلط به زبان برنامه‌نویسی پایتون (PYTHON)}{تسلط نسبی بر مباحث سری زمانی در مالی }
	

\cvlistdoubleitem[\ding{72}\ding{72}\ding{78}\ding{73}]{آشنایی با زبان‌های \lr{(C, Matlab)}}{آشنایی کامل با ریاضیات مالی}			



\cvlistdoubleitem[\ding{72}\ding{72}\ding{73}\ding{73}]{آشنایی با مباحث بهینه‌سازی \lr{(Optimization)}}{آشنایی با هوش مصنوعی و مباحث یادگیری عمیق}	
		
\section{استعداد و زمینه‌های کاری مورد علاقه}
\cvline{}{مهندسی مالی، مدیریت ریسک ، قیمت‌گذاری اختیارات، معاملات الگوریتمی، یادگیری ماشین، یادگیری عمیق.}

\end{document}


























%	\section{مهارت‌ها}
%	\cvline{}{در زمینهٔ زبانهای برنامه‌نویسی و نرم‌افزارهای زیر تجربیاتی دارم:}
%	\cvline{زبانهای برنامه‌نویسی}{}
%	\begin{flushleft}
%		\begin{latin}
%			{R, C and C++.}
%		\end{latin}
%	\end{flushleft}
%	\cvline{نرم‌افزارها و بسته‌ها}{}
%	\begin{flushleft}
%		\begin{latin}
%			MATLAB, Stata, EViews, \LaTeX, Microsoft Office (Word, Excel, Access,  PowerPoint), TeXMaker, Google colab,\XePersian, Farsi\TeX, Bib\TeX, MiKTeX, and Some Others.
%		\end{latin}
%	\end{flushleft}
%	
%	
%	
%	\section{عضویت در گروه‌ها}
%	\begin{latin}
%		\begin{flushleft}
%			{\small • Member of the stackoverflow. \hfill {\scriptsize\em https://stackoverflow.com/}}\\
%			{\small •  member of Cross Validated usergroup. \hfill {\scriptsize\em https://stats.stackexchange.com/}}\\
%			{\small • Member of the Iranian \LaTeX{} usergroup. \hfill {\scriptsize\em http://www.parsilatex.com}}\\
%			{\small • Member of the TeX LaTeX usergroup \hfill {\scriptsize\em https://tex.stackexchange.com/}}\\
%		\end{flushleft}
%	\end{latin}
%	
%	
%	%\cvline{زبان انگلیسی}{در زمینه‌ی زبان انگلیسی هم در سطح مناسبی قرار دارم.}
%	%%%%%%%%%%%%%%%%%{تألیفات}
%	
%	
%	
%	
%	%
%	%\section{کارگاههایی که شرکت کرده‌ام}
%	%\cvline{۱۳۸۶}{شبیه‌سازی مونت کارلو، {\small دانشگاه صنعتی شریف}}
%	%\cvline{۱۳۸۳}{شبکه‌های عصبی، {\small دانشگاه تربیت معلم سبزوار}}
%	%\cvline{۱۳۸۲}{دوره آموزشی روش تحقیق در علوم پایه، {\small دانشگاه تربیت معلم سبزوار}}
%	%
%	%\section{سایر موارد}
%	%\cvlistitem{	دارای کارت پایان خدمت}
%	%\cvlistitem{	یکی از دانشجویان منتخب دکترا در دانشگاه علم و صنعت ایران در سال 1385 به عنوان دانشجوی نخبه با معدل 18.75.}
%	%\cvlistitem{	بالاترین امتیاز در بین اعضای هیأت علمی گروه ریاضی دانشگاه تربیت معلم سبزوار در دو ترم بر طبق ارزشیابی اساتید از دانشجویان.}
%	%\cvlistitem{دارای هوش هیجانی}

